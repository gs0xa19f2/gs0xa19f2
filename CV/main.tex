%-------------------------
% Resume in Latex
% Author : Adapted for Sergey Gusev (gs0xa19f2)
% Based off of: https://github.com/sb2nov/resume
% License : MIT
%------------------------

\documentclass[letterpaper,12pt]{article}

\usepackage{latexsym}
\usepackage[empty]{fullpage}
\usepackage{titlesec}
\usepackage{marvosym}
\usepackage[usenames,dvipsnames]{color}
\usepackage{verbatim}
\usepackage{enumitem}
\usepackage[hidelinks]{hyperref}
\usepackage{fancyhdr}
\usepackage[russian]{babel}
\usepackage{tabularx}
\input{glyphtounicode}

\pagestyle{fancy}
\fancyhf{}
\fancyfoot{}
\renewcommand{\headrulewidth}{0pt}
\renewcommand{\footrulewidth}{0pt}

\addtolength{\oddsidemargin}{-0.5in}
\addtolength{\evensidemargin}{-0.5in}
\addtolength{\textwidth}{1in}
\addtolength{\topmargin}{-.5in}
\addtolength{\textheight}{1.0in}

\urlstyle{same}

\raggedbottom
\raggedright
\setlength{\tabcolsep}{0in}

% Увеличенные отступы между секциями и над именем
\titleformat{\section}{
  \vspace{7pt}\scshape\raggedright\large % <--- Изменено здесь
}{}{0em}{}[\color{black}\titlerule \vspace{5pt}] % <--- и здесь

\pdfgentounicode=1

%-------------------------
% Custom commands
\newcommand{\resumeItem}[1]{
  \item\small{
    {#1 \vspace{-2pt}}
  }
}

\newcommand{\resumeSubheading}[4]{
  \vspace{-2pt}\item
    \begin{tabular*}{0.97\textwidth}[t]{l@{\extracolsep{\fill}}r}
      \textbf{#1} & #2 \\
      \textit{\small#3} & \textit{\small #4} \\
    \end{tabular*}\vspace{-7pt}
}

\newcommand{\resumeSubSubheading}[2]{
    \item
    \begin{tabular*}{0.97\textwidth}{l@{\extracolsep{\fill}}r}
      \textit{\small#1} & \textit{\small #2} \\
    \end{tabular*}\vspace{-7pt}
}

\newcommand{\resumeProjectHeading}[2]{
    \item
    \begin{tabular*}{0.97\textwidth}{l@{\extracolsep{\fill}}r}
      \small#1 & #2 \\
    \end{tabular*}\vspace{-7pt}
}

\newcommand{\resumeSubItem}[1]{\resumeItem{#1}\vspace{-4pt}}

\renewcommand\labelitemii{$\vcenter{\hbox{\tiny$\bullet$}}$}

\newcommand{\resumeSubHeadingListStart}{\begin{itemize}[leftmargin=0.15in, label={}]}
\newcommand{\resumeSubHeadingListEnd}{\end{itemize}}
\newcommand{\resumeItemListStart}{\begin{itemize}}
\newcommand{\resumeItemListEnd}{\end{itemize}\vspace{-5pt}}

%-------------------------------------------
%%%%%%  RESUME STARTS HERE  %%%%%%%%%%%%%%%%%%%%%%%%%%%%

\begin{document}

%----------HEADING----------
\begin{center}
    {\Huge \scshape Сергей Гусев} \\[16pt] % Увеличенный вертикальный отступ под именем
    \href{mailto:greygosling6@gmail.com}{greygosling6@gmail.com} $|$ 
    +7 920 120 6928 $|$
    \href{https://github.com/gs0xa19f2}{github.com/gs0xa19f2} $|$
    \href{https://t.me/gs0xa19f2}{@gs0xa19f2} $|$
    \href{https://leetcode.com/gs0xa19f2}{leetcode.com/gs0xa19f2}
\end{center}

%-----------ABOUT-----------
\section{О себе}
Я студент третьего курса направления «Компьютерные технологии: программирование и искусственный интеллект» в Университете ИТМО. Увлечен Java-разработкой, Kotlin, Go, алгоритмами и созданием эффективных решений на различных платформах

%-----------EDUCATION-----------
\section{Образование}
  \resumeSubHeadingListStart
    \resumeSubheading
      {Университет ИТМО}{} 
      {Бакалавр, Компьютерные технологии: программирование и искусственный интеллект}{}
    \resumeSubheading
      {Лицей научно-инженерного профиля (ЛНИП)}{}
      {Среднее общее образование}{}
  \resumeSubHeadingListEnd

%-----------PROJECTS-----------
\section{Проекты}
    \resumeSubHeadingListStart
      \resumeProjectHeading
          {\href{https://github.com/gs0xa19f2/ITMO/tree/main/Kotlin/ChatBotDSL}{\textbf{DSL ChatBot}} $|$ \emph{Kotlin}}{}
          \resumeItemListStart
            \resumeItem{DSL для создания и управления функциональностью чат-ботов, включая обработку сообщений и управление контекстами}
          \resumeItemListEnd

      \resumeProjectHeading
          {\href{https://github.com/gs0xa19f2/ITMO/tree/main/Kotlin/MultiplatformClient}{\textbf{Мультиплатформенный HTTP-клиент}} $|$ \emph{Kotlin (JVM, JS, Native)}}{}
          \resumeItemListStart
            \resumeItem{Поддерживает JVM, JS (Node.js, Browser) и Native платформы, с общей логикой и платформо-специфическими реализациями}
          \resumeItemListEnd

      \resumeProjectHeading
          {\href{https://github.com/gs0xa19f2/Hyperskill/tree/main/Projects/Go/Version\%20Control\%20System\%20}{\textbf{Система контроля версий}} $|$ \emph{Go}}{}
          \resumeItemListStart
            \resumeItem{Предоставляет основные функции Git}
          \resumeItemListEnd

      \resumeProjectHeading
          {\href{https://github.com/gs0xa19f2/Hyperskill/tree/main/Projects/Go/Regex\%20Engine\%20}{\textbf{Regex Engine}} $|$ \emph{Go}}{}
          \resumeItemListStart
            \resumeItem{Движок для работы с регулярными выражениями, способный анализировать и сопоставлять шаблоны в тексте}
          \resumeItemListEnd
    \resumeSubHeadingListEnd

%-----------TECHNICAL SKILLS-----------
\section{Технические навыки}
 \begin{itemize}[leftmargin=0.15in, label={}]
    \small{\item{
     \textbf{Языки программирования}{: Java, Kotlin, Go, Python, C/C++, SQL, Bash} \\
     \textbf{Фреймворки и технологии}{: Spring, Docker, Git, JUnit, REST, MongoDB, Linux} \\
    }}
 \end{itemize}

%-----------OTHER-----------
\section{Прочее}
 \begin{itemize}[leftmargin=0.15in, label={}]
    \item{\href{https://github.com/gs0xa19f2/gs0xa19f2/blob/main/Рекомендательное%20письмо%20UCSD.pdf}{Получено рекомендательное письмо от профессоров \textbf{А.С. Куликова} (СПбГУ, ПОМИ РАН) и \textbf{П. Певзнера} (UCSD, University of California, San Diego).}}
    \item{\href{https://hyperskill.org/wrapped/year-2024/398153840?utm_source=wrapped_hs&utm_medium=social&utm_campaign=newyear_wrapped2024}{Вошёл в топ 3\% студентов Hyperskill (JetBrains Academy) по всему миру по итогам 2024 года.}}
 \end{itemize}

%-------------------------------------------
\end{document}
