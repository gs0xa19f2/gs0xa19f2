%-------------------------
% Resume in Latex
% Author : Adapted for Sergei Gusev (gs0xa19f2)
%------------------------

\documentclass[letterpaper,12pt]{article}

\usepackage{latexsym}
\usepackage[empty]{fullpage}
\usepackage{titlesec}
\usepackage{marvosym}
\usepackage[usenames,dvipsnames]{color}
\usepackage{verbatim}
\usepackage{enumitem}
\usepackage[hidelinks]{hyperref}
\usepackage{fancyhdr}
\usepackage[russian]{babel}
\usepackage{tabularx}
\input{glyphtounicode}

\pagestyle{fancy}
\fancyhf{}
\fancyfoot{}
\renewcommand{\headrulewidth}{0pt}
\renewcommand{\footrulewidth}{0pt}

\addtolength{\oddsidemargin}{-0.5in}
\addtolength{\evensidemargin}{-0.5in}
\addtolength{\textwidth}{1in}
\addtolength{\topmargin}{-.5in}
\addtolength{\textheight}{1.0in}

\urlstyle{same}

\raggedbottom
\raggedright
\setlength{\tabcolsep}{0in}

\titleformat{\section}{
  \vspace{7pt}\scshape\raggedright\large
}{}{0em}{}[\color{black}\titlerule \vspace{5pt}]

\pdfgentounicode=1

%-------------------------
% Custom commands
\newcommand{\resumeItem}[1]{
  \item\small{{#1 \vspace{-2pt}}}
}

\newcommand{\resumeSubheading}[4]{
  \vspace{-2pt}\item
    \begin{tabular*}{0.97\textwidth}[t]{l@{\extracolsep{\fill}}r}
      \textbf{#1} & #2 \\
      \textit{\small#3} & \textit{\small #4} \\
    \end{tabular*}\vspace{-7pt}
}

\newcommand{\resumeProjectHeading}[2]{
    \item
    \begin{tabular*}{0.97\textwidth}{l@{\extracolsep{\fill}}r}
      \small#1 & #2 \\
    \end{tabular*}\vspace{-7pt}
}

\newcommand{\resumeSubHeadingListStart}{\begin{itemize}[leftmargin=0.15in, label={}]}
\newcommand{\resumeSubHeadingListEnd}{\end{itemize}}
\newcommand{\resumeItemListStart}{\begin{itemize}[leftmargin=0.32in, label={}]}
\newcommand{\resumeItemListEnd}{\end{itemize}\vspace{-5pt}}

%-------------------------------------------
%%%%%%  RESUME STARTS HERE  %%%%%%%%%%%%%%%%%%%%%%%%%%%%

\begin{document}

\begin{center}
    {\Huge \scshape Сергей Гусев} \\[8pt]
    {\Large \scshape Software Engineer} \\[12pt]
    \href{mailto:greygosling6@gmail.com}{greygosling6@gmail.com} $|$ 
    +7 920 120 6928 $|$
    \href{https://github.com/gs0xa19f2}{github.com/gs0xa19f2} $|$
    \href{https://t.me/gs0xa19f2}{@gs0xa19f2} $|$
    \href{https://leetcode.com/gs0xa19f2}{leetcode.com/gs0xa19f2}
\end{center}

%-----------EXPERIENCE-----------
\section{Опыт}
\resumeSubHeadingListStart

  \resumeSubheading
    {Picodata}{2025 -- наст.вр.}
    {Rust-разработчик}{}
    \resumeItemListStart
        \resumeItem{\textit{Picodata — распределенная in-memory NewSQL СУБД с поддержкой плагинов на Rust }}
        
        \resumeItem{\textbf{Развитие инструмента для разработки и управления плагинами Picodata}}
            \resumeItemListStart
                \resumeItem{Самостоятельно реализовал ключевые улучшения, которые легли в основу релизов \textbf{2.9.0} и \textbf{3.0.0}, упростив и ускорив процессы разработки для всех команд. }
                \resumeItem{Ключевые доработки: расширение шаблона плагина, валидация конфигураций, стандартизация именования артефактов, добавление опций для гибкой сборки, тестирования на разных версиях Picodata, гибкого применения конфигов и поддержку внешних плагинов в топологии}
            \resumeItemListEnd

        \resumeItem{\textbf{Повышение совместимости в протоколе Cassandra поверх Picodata}}
            \resumeItemListStart
                \resumeItem{Устранил критические баги в парсере CQL, что позволило корректно обрабатывать запросы от нативных драйверов и сторонних инструментов.}
                \resumeItem{Решенные проблемы включали поддержку типов данных в разном регистре, использование зарезервированных слов в качестве имен, обработку сложных запросов и поддержку совместного использования команд, согласованного с Cassandra}
            \resumeItemListEnd
            
        \resumeItem{\textbf{Проектирование API в low-code платформе для Финтех-партнера}}
            \resumeItemListStart
                \resumeItem{Спроектировал и реализовал новый RESTful API для усечения таблиц, заменив устаревший механизм с отдельной авторизацией после миграции на Picodata. }
                \resumeItem{Обеспечил консистентное выполнение операции в распределенном кластере через асинхронную отправку RPC-запросов на все реплики}
            \resumeItemListEnd
    \resumeItemListEnd 

\resumeSubHeadingListEnd

%-----------SKILLS-----------
\section{Навыки}
 \resumeSubHeadingListStart
    \resumeItem{\textbf{Технологии и фреймворки:} Spring Boot, Microservices, Kafka, Docker, REST, gRPC, JUnit, Mockito, Grafana, Observability, Git, Linux}
    \resumeItem{\textbf{Языки программирования:} Rust, Kotlin, Go, C/C++, Java, Python, Bash}
    \resumeItem{\textbf{Базы данных:} Picodata, Cassandra, MongoDB, MySQL, PostgreSQL}
    \resumeItem{\textbf{Языки:} Русский, Английский - C1}
 \resumeSubHeadingListEnd

%-----------PROJECTS-----------
\section{Проекты}
\resumeSubHeadingListStart
  \resumeProjectHeading
    {\textbf{Плагин кеширования и поиска стран для Picodata}}{}
    \resumeItemListStart
      \resumeItem{HTTP API с кешированием (TTL, автообновление), обработкой ошибок и интеграционными тестами}
    \resumeItemListEnd

  \resumeProjectHeading
    {\textbf{Библиотека команд, аудита и метрик для серверного приложения}}{}
    \resumeItemListStart
      \resumeItem{Стартер для Spring Boot: очередь команд с приоритезацией и валидацией, AOP-аудит действий, метрики, глобальная обработка ошибок, интеграция с Kafka, unit \& integration тесты}
    \resumeItemListEnd
\resumeSubHeadingListEnd

%-----------EDUCATION-----------
\section{Образование}
\resumeSubHeadingListStart
  \resumeSubheading
    {Университет ИТМО}{} 
    {Студент 3 курса, Компьютерные технологии: программирование и искусственный интеллект}{}
\resumeSubHeadingListEnd

%-----------ACHIEVEMENTS-----------
\section{Достижения}
 \begin{itemize}[leftmargin=0.15in, label={}]
    \item{\href{https://github.com/gs0xa19f2/gs0xa19f2/blob/main/%D0%A0%D0%B5%D0%BA%D0%BE%D0%BC%D0%B5%D0%BD%D0%B4%D0%B0%D1%82%D0%B5%D0%BB%D1%8C%D0%BD%D0%BE%D0%B5%20%D0%BF%D0%B8%D1%81%D1%8C%D0%BC%D0%BE%20UCSD.pdf}{Рекомендательное письмо от профессоров \textbf{А.\,С. Куликова} (СПбГУ, ПОМИ РАН) и \textbf{П. Певзнера} (UCSD).}}
    \item{\href{https://hyperskill.org/wrapped/year-2024/398153840?utm_source=wrapped_hs&utm_medium=social&utm_campaign=newyear_wrapped2024}{Топ 3\% студентов Hyperskill (JetBrains Academy) по итогам 2024 года.}}
    \item{\href{https://github.com/gs0xa19f2/gs0xa19f2/blob/main/Результаты%20тренировок%20по%20алгоритмам%20(ноябрь%202025).xlsx}{Топ-30 рейтинга тренировок по алгоритмам Яндекса по итогам 2025 года.}}
 \end{itemize}

%-------------------------------------------
\end{document}
