%%%%%%%%%%%%%%%%%%%%%%%%%%%%%%%%%%%%%%%%%
% Developer CV - XeLaTeX Version for Cyrillic
% Based on LaTeX Template Version 1.1
% https://www.LaTeXTemplates.com
% Adapted for gs0xa19f2 on 2025-04-25
%%%%%%%%%%%%%%%%%%%%%%%%%%%%%%%%%%%%%%%%%

\documentclass[10pt]{developercv} % Main Font size 10pt

%----------------------------------------------------------------------------------------
%	PACKAGES AND FONT CONFIGURATION for XeLaTeX
%----------------------------------------------------------------------------------------

\usepackage[russian]{babel} % For Russian hyphenation, date formats, etc.

\usepackage{fontspec}       % Primary package for font selection in XeLaTeX
\setmainfont{Liberation Serif} % Set main font (supports Cyrillic)
% You can try others like: {PT Serif}, {DejaVu Serif}, {Times New Roman}

% Set sans-serif and monospace fonts if needed by the template elements
\setsansfont{Liberation Sans} % Companion sans-serif font
\setmonofont{Liberation Mono} % Companion monospace font for \texttt{}

\usepackage{hyperref}       % For clickable links (load relatively late)
% Ensure hyperref handles UTF-8 characters in URLs and metadata correctly with XeLaTeX
\hypersetup{              % Configure hyperref colors/metadata if desired
    pdfencoding=auto, % Usually good for XeLaTeX
    unicode=true,    % Enable Unicode characters in bookmarks
    pdftitle={Резюме Сергея Гусева},
    pdfauthor={Сергей Гусев},
    pdfcreator={XeLaTeX with developercv},
    colorlinks=true,
    linkcolor=black, % Color for internal links (e.g., ToC - not used here)
    urlcolor=blue,   % Color for URLs, choose a color you like (e.g., blue, black)
    citecolor=black  % Color for citations (not used here)
}

%----------------------------------------------------------------------------------------

\begin{document}

%----------------------------------------------------------------------------------------
%	TITLE AND CONTACT INFORMATION
%----------------------------------------------------------------------------------------

\begin{minipage}[t]{0.45\textwidth} % Left column: Name and Title
	\vspace{-\baselineskip} % Vertical alignment adjustment

	% Using \textbf instead of \MakeUppercase for better Cyrillic compatibility with some fonts
	\colorbox{black}{{\HUGE\textcolor{white}{\textbf{СЕРГЕЙ}}}} % First name
	\colorbox{black}{{\HUGE\textcolor{white}{\textbf{ГУСЕВ}}}}   % Last name

	\vspace{6pt} % Vertical space

	{\huge Java-разработчик} % Job Title
\end{minipage}
\hfill % Horizontal space
\begin{minipage}[t]{0.27\textwidth} % Center column: Contact Info (Width 0.27)
	\vspace{-\baselineskip} % Vertical alignment adjustment
    \small % <-- Set font size to small (approx 9pt) for this block
	\icon{At}{12}{\href{mailto:greygosling6@gmail.com}{greygosling6@gmail.com}}\\
	\icon{Phone}{12}{\href{tel:+79201206928}{+7 920 120 6928}}\\
\end{minipage}
\hfill % Horizontal space
\begin{minipage}[t]{0.27\textwidth} % Right column: Contact Info (Width 0.27)
	\vspace{-\baselineskip} % Vertical alignment adjustment
    \small % <-- Set font size to small (approx 9pt) for this block
	\icon{Github}{12}{\href{https://github.com/gs0xa19f2}{github.com/gs0xa19f2}}\\
	\icon{TelegramPlane}{12}{\href{https://t.me/gs0xa19f2}{@gs0xa19f2}}
\end{minipage}

\vspace{0.5cm} % Vertical space after header

%----------------------------------------------------------------------------------------
%	WHO AM I / INTRODUCTION & SKILLS
%----------------------------------------------------------------------------------------

\cvsect{Кто я?} % Section Title

\begin{minipage}[t]{0.4\textwidth} % Left column: Introduction Text
	\vspace{-\baselineskip} % Vertical alignment adjustment
	Я студент третьего курса направления "Компьютерные технологии: программирование и искусственный интеллект" в Университете ИТМО. Увлечен Kotlin, Java-разработкой, алгоритмами и созданием эффективных решений на различных платформах.
    Посетите мой Github, если хотите узнать больше.
\end{minipage}
\hfill % Horizontal space
\begin{minipage}[t]{0.5\textwidth} % Right column: Programming Languages Bar Chart
	\vspace{-\baselineskip} % Vertical alignment adjustment

	% --- PROGRAMMING LANGUAGES BAR CHART ---
	\begin{barchart}{5.5} % Max bar width in cm
		\baritem{Java}{100}
        \baritem{Kotlin}{90}
		\baritem{Go}{80}
        \baritem{Python}{70}
        \baritem{C/C++}{60}
	\end{barchart}
\end{minipage}

% --- TOOLS & FRAMEWORKS BUBBLES ---
\begin{center}
    % Размеры немного уменьшены, SQL сокращен для лучшего вида
	\bubbles{5/Docker, 5/Spring, 5/Git, 4/SQL, 3/MongoDB, 4/Linux, 4/Bash}
\end{center}

%----------------------------------------------------------------------------------------
%	PROJECTS
%----------------------------------------------------------------------------------------

\cvsect{Проекты} % Section Title

\begin{entrylist}
	\entry
		{} % Date (optional)
		{DSL ChatBot (Kotlin)} % Project Title
		{\href{https://github.com/gs0xa19f2/ITMO/tree/main/Kotlin/ChatBotDSL}{Ссылка на проект}} % Link/Context
		{Разработан DSL для создания и управления функциональностью чат-ботов, включая обработку сообщений и управление контекстами.}
	\entry
		{} % Date (optional)
		{Мультиплатформенный HTTP-клиент (Kotlin)} % Project Title
		{\href{https://github.com/gs0xa19f2/ITMO/tree/main/Kotlin/MultiplatformClient}{Ссылка на проект}} % Link/Context
		{Создан HTTP-клиент с поддержкой JVM, JS (Node.js, Browser) и Native платформ, с общей логикой и платформо-специфическими реализациями.}
    \entry
		{} % Date (optional)
		{Система контроля версий (Go)} % Project Title
		{\href{https://github.com/gs0xa19f2/Hyperskill/tree/main/Projects/Go/Version\%20Control\%20System\%20}{Ссылка на проект}} % Link/Context
		{Реализована легковесная система контроля версий с нуля, предоставляющая основные функции Git.}
    \entry
		{} % Date (optional)
		{Regex Engine (Go)} % Project Title
		{\href{https://github.com/gs0xa19f2/Hyperskill/tree/main/Projects/Go/Regex\%20Engine\%20}{Ссылка на проект}} % Link/Context
		{Написан собственный движок для работы с регулярными выражениями, способный анализировать и сопоставлять шаблоны в тексте.}
\end{entrylist}

%----------------------------------------------------------------------------------------
%	EDUCATION
%----------------------------------------------------------------------------------------

\cvsect{Образование} % Section Title

\begin{entrylist}
	\entry
		{} % Dates (optional)
		{Университет ИТМО} % Institution Name
		{Санкт-Петербург}
		{} % Description (optional)
	\entry
		{} % Dates (optional)
		{Лицей научно-инженерного профиля (ЛНИП)} % Institution Name
		{г. Королёв, Московская область} % Location/Details
		{} % Description (optional)
\end{entrylist}

%----------------------------------------------------------------------------------------
%	OTHER
%----------------------------------------------------------------------------------------

\cvsect{Другое} % Section Title

% Текст идет прямо под заголовком секции
Получено рекомендательное письмо от профессоров \textbf{Александра С. Куликова} (СПбГУ, ПОМИ РАН) и \textbf{Павла Певзнера} (UCSD) из \textbf{Университета Калифорнии в Сан-Диего}.
\href{https://github.com/gs0xa19f2/gs0xa19f2/blob/main/Рекомендательное%20письмо%20UCSD.pdf}{\textbf{Нажмите здесь для просмотра письма.}}

\vspace{0.3cm} % Добавляем небольшой отступ перед следующим пунктом

Вошел в \textbf{топ 3\% студентов Hyperskill (JetBrains Academy) по всему миру} по итогам 2024 года.
\href{https://hyperskill.org/wrapped/year-2024/398153840?utm_source=wrapped_hs&utm_medium=social&utm_campaign=newyear_wrapped2024}{\textbf{Подробнее здесь.}} % Добавлена информация о Hyperskill

%----------------------------------------------------------------------------------------

\end{document}